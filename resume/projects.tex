\es{\cvsection{Prácticas, becas y proyectos}}
\en{\cvsection{Projects}}

\begin{cventries}
\es{\cventry
    {Prácticas}
    {Prácticas en GRASIA UCM}
    {Madrid}
    {Sep 2021-Dic 2021}
    {
    % \begin{cvitems}
    %     \item Iniciación a los trabajos realizados en GRASIA (véase Experiencia laboral)
    % \end{cvitems}
    }
    \vspace{-10pt}
}

\en{\cventry
    {Apprenticeship}
    {Apprenticeship at GRASIA UCM}
    {Madrid}
    {Sep 2021-Dec 2021}
    {
    % \begin{cvitems}
    %     \item Prior to working at GRASIA (see Work Experience)
    % \end{cvitems}
    }
    \vspace{-10pt}
}

\es{\cventry
    {Participante}
    {Estudiante de Google Summer of Code 2021 con Haskell.org}
    {Remoto}
    {Verano 2021}
    {
    \begin{cvitems}
        \item Programación en Haskell arreglando el módulo \texttt{ihaskell-widgets} del kernel IHaskell para Jupyter.
        \item Uso de tipos dependientes con la librería \texttt{Singletons}
    \end{cvitems}
    }
}

\en{\cventry
    {Participant}
    {2021 Google Summer of Code student w/ Haskell.org}
    {Remote}
    {Summer 2021}
    {
    \begin{cvitems}
        \item Programming in Haskell fixing the module \texttt{ihaskell-widgets} from Jupyter's kernel IHaskell.
        \item Working with dependent types using the \texttt{Singletons} library
    \end{cvitems}
    }
}

\es{
\cventry
    {Desarrollador}
    {JustGetMyData.com}
    {Madrid}
    {2020-Act}
    {
        \begin{cvitems}
            \item Web escrita en Jekyll y montaje con GitHub Pages
            \item Menciones en decenas de medios (Motherboard de Vice, Les Numériques, Protonmail...)
        \end{cvitems}
    }
}
\en{
\cventry
    {Developer}
    {JustGetMyData.com}
    {Madrid}
    {2020-Curr.}
    {
        \begin{cvitems}
            \item Web with Jekyll on GitHub Pages
            \item As seen about in media (Vice's Motherboard, Les Numériques, Protonmail...)
        \end{cvitems}
    }
}

\ifPersonalPage
\es{
\cventry
    {Developer}
    {Página personal}
    {Madrid}
    {2014-Act}
    {
    \begin{cvitems}
        \item Diseño ``desde cero'' usando Bootstrap
        \item Backend en Symfony
        \item Internacionalización
    \end{cvitems}
    }
}

\en{\cventry
    {Developer}
    {Personal Webpage}
    {Madrid}
    {2014-Curr.}
    {\begin{cvitems}
        \item Design without templates, Back-end on Symfony
        \item Localisation
    \end{cvitems}}}
\fi

\es{
\cventry
    {Desarrollador 3rd party}
    {Simkl TV Tracker}
    {Remoto}
    {2016 - 2019}
    {
        \begin{cvitems}
            \item Creación de pugins que marquen automáticamente la película o serie como `Vista' en Simkl
            \item Versión para Kodi escrita en Python3 (2016 - 2017)
            \item Versión para Emby escrita en .NET C\# (2018 - 2019)
        \end{cvitems}
    }
}

\en{
\cventry
    {3rd Party Developer}
    {Simkl TV Tracker}
    {Remote}
    {2016-2019}
    {
    \begin{cvitems}
        \item Creation of add-ons to auto-scrobble a TV Show or Movie to Simkl
        \item Kodi version on Python3 (2016-2017) and Emby version on C\# (2018-2019)
    \end{cvitems}
    }
}

\iffalse
\es{    
\cventry
    {Desarrollador Fullstack}
    {Cuentos y Poemas por Teléfono}
    {Madrid}
    {2016 - Act}
    {
        \begin{cvitems}
            \item Infraestructura en AWS usando E2 (CentOS + Apache) y RDS (MariaDB)
            \item Frontend en Bootstrap y API JSON para bot de Telegram. Backend en Symfony.
            \item Dashboard con estadísticas de los datos usando Plotly.js
        \end{cvitems}
    }
}
\en{
\cventry
    {Fullstack Developer}
    {Tales and Poems over the Phone}
    {Madrid}
    {2016 - Curr.}
    {
    \begin{cvitems}
        \item AWS infrastructure using E2 (CentOS + Apache) and RDS (MariaDB)
        \item Frontend on Bootstrap and Telegram Bot. Symfony backend.
        \item Dashboard with various data statistics using Plotly.js
    \end{cvitems}
    }
}
\fi

    
\es{\cventry
    {Desarrollador}
    {InvProy $\alpha$: Un Simulador de Redes por y para alumnos}
    {Madrid}
    {Sep 2015 - Feb 2017}
    {
    \begin{cvitems}
        \item Proyecto de Investigación parte del programa de «Bachillerato de Excelencia»
        \item Código del proyecto en Python3, interfaz en Gtk+, Memoria escrita en LaTeX
        \item Ganador del II Encuentro Preuniversitario Complutense de Jóvenes Investigadores (2017)
    \end{cvitems}
    }}
    
\en{\cventry
    {Developer}
    {InvProy $\alpha$: A Network Simulator for and by students}
    {Madrid}
    {Sep 2015 - Feb 2017}
    {
    \begin{cvitems}
        \item Part of the "Baccalaureate of Excellence" program
        \item Code on Python3, Gtk+ GUI, report written in LaTeX
        \item Winner of the 2nd Pre-grad Meeting of Young Researchers at UCM (2017)
    \end{cvitems}
    }}
\end{cventries}